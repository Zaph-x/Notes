\section{Komplekse tal}
Komplekse tal, er tal der ikke nødvendigvis kan forklares. 
Disse tal indeholder blandt andet tal som $i$, $j$ og $z$, og beskrives med domænet $\mathbb{C}$.
Tallet i bruges blandt andet til at forklare ligningen
\begin{equation}
    \label{eqn:complex}
    x^2=-1
\end{equation}
Med ligningen i \cref{eqn:complex} kan $16i^2=-16$ nu udregnes. 
Med dette kan det konkluderes at $4i$ er løsningen til $x^2=-16$, da $(4i)^2=-16$.
Ligeledes kan ligningen $(x-1)^2=-16$, løses ved $x-1=4i$, hvilket giver at $x=1+4i$.
Derfor kan det konkluderes at alle komplekse tal har en reel del og et imaginær del.
