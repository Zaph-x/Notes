Et domæne er et en samling af værdier med operatorerne plus og gange.
Eksempelvis er $\mathrm{R}$ et domæne, ligesom $ \mathrm{N}$ er et domæne.
Dette kapitel vil dække domæner og deres egenskaber.

Disse domæner kan sidestilles med OOP, hvor en klasse kan indeholde nogle bestemte properties.
\begin{table}[h]
    \centering
    \begin{tabular}{|l|r|}
        \hline
        Domæne & Indhold\\\hline
        $\mathbb{R}$ & Alle reelle tal $\set*{\dots,-2,-1.816,0,1,\sqrt{2},\pi,\dots}$\\
        $\mathbb{C}$ & Komplekse tal $\set*{i,j,z}$\\
        $\mathbb{N}$ & Naturlige tal $\set*{1,2,3,\dots}$\\
        $\mathbb{Z}$ & Hele tal $\set*{\dots,-3,-2,-1,0,1,2,3,\dots}$\\
        $GF(2)$ & Et domæne der kun indeholder 0 og 1\\
        \hline
    \end{tabular}
    \label{tab:fields}
    \caption{Tabel over nogle af de eksisterende domæner.}
\end{table}

\section{Komplekse tal}
Komplekse tal, er tal der ikke nødvendigvis kan forklares. 
Disse tal indeholder blandt andet tal som $i$, $j$ og $z$, og beskrives med domænet $\mathbb{C}$.
Tallet i bruges blandt andet til at forklare ligningen
\begin{equation}
    \label{eqn:complex}
    x^2=-1
\end{equation}
Med ligningen i \cref{eqn:complex} kan $16i^2=-16$ nu udregnes. 
Med dette kan det konkluderes at $4i$ er løsningen til $x^2=-16$, da $(4i)^2=-16$.
Ligeledes kan ligningen $(x-1)^2=-16$, løses ved $x-1=4i$, hvilket giver at $x=1+4i$.
Derfor kan det konkluderes at alle komplekse tal har en reel del og et imaginær del.

\section{Abstraktion af domæner}
Abstraktion kan være en nødvændighed i matematik, og ikke mindst programmering.
Eksempelvis kan ligningen $ax+b=c$, hvor $a$ ikke er nul.
Dette kan gøres programmatisk på følgende vis:
\lstinputlisting[
	language=Python,
    label=lst:python_abstraction,
	caption={Python kode til at udregne $ax+b=c$, hvor $a$ ikke er nul.\\Koden kan findes i ./labs/fields/abstraction.py}
]{labs/fields/abstraction.py}
Som det også fremkommer af \cref{lst:python_abstraction} kan komplekse tal også bruges i python.
Disse komplekse tal skal dog noteres som $j$.

\section{GF(2)}
$GF(2)4$ er kort for \textit{Galois Field 2}. 
$GF(2)$ er relativt nemt at forklare, da dette kun dækker talene 1 og 0.
Derfor kan aritmetikken i $GF(2)$ også nemt dækkes.
Aritmetikken kan opsummeres i de følgende to tabeller:
\begin{table}[h]
    \centering
    \begin{tabular}{c|cc}
        $\times$&0&1\\\hline
        0&0&0\\
        1&0&1
    \end{tabular}
    \label{tab:fields}
    \caption{Multiplikation i $GF(2)$}
\end{table}
\begin{table}[h]
    \centering
    \begin{tabular}{c|cc}
        +&0&1\\\hline
        0&0&1\\
        1&1&0
    \end{tabular}
    \label{tab:fields}
    \caption{Addition i $GF(2)$}
\end{table}
Addition i $GF(2)$ foregår i modulo 2.
Multiplikation foregår præcis som normal Multiplikation.
Dette fremkommer også af \cref{lst:gf2lab}
\lstinputlisting[
	language=Python,
    label=lst:gf2lab,
	caption={Python kode til udregning af $GF(2)$.\\Koden kan findes i ./labs/fields/gf2lab.py}
]{labs/fields/gf2lab.py}
\section{Practice Problems}
Dette afsnit vil behandle de givne opgaver fra forelæsnignerne til kapitlet.
\subsection{Problem 1.7.13}
Kursusgang 4.\\
For each of the following problems, calculate the answe over $GF(2)$
\begin{itemize}
    \item \begin{equation}
        1+1+1+0
    \end{equation}
    Ovenstående udtryk evalueres som $1$, da $3\:\mathbf{mod}\:2=1$.
    \item \begin{equation}
        1\cdot1+0\cdot1+0\cdot0+1\cdot1
    \end{equation}
    Ovenstående udtryk evalueres som $0$, da de aritmetiske grundprincipper stadig gælder, og dermed også regnehierakiet.
    Dette betyder at udtrykket skal læses som
    \begin{equation}
        1+0+0+1
    \end{equation}
    Og dermed giver det $0$, da $2\:\mathbf{mod}\:2=0$.
    \item \begin{equation}
        (1+1+1)\cdot(1+1+1+1)
    \end{equation}
    Ovenstående udtryk evalueres som $0$, da udtrykket evalueres som $1\cdot0$, da der regnes modulo 2.
\end{itemize}