\begin{frdef}[Vektorer]
	En vektor er i geometrien et objekt, der er defineret ved at have en længde og en retning.
\end{frdef}

\todo{Vend tilbage til dette. Dette er kun en bid af noten}


124-144, 149-167

\section{Geometrien i sæt af vektorer}
Dette afsnit har til formål at beskrive det geometriske aspekt af vektorer. Det vil dække spænding af vektorer. Dette afsnit vil beskrive todimensionale vektorer som 2-vektorer, og tredimensionelle vektorer som 3-vektorer. Der vil blive brugt bracket notation for vektorer.

\subsection{Spændinger af vektorer over $\mathbb{R}$}
 
\begin{equation}
	\mathrm{Span}\{v\}=\set*{\alpha v; \alpha\in\mathbb{R}}
\end{equation}

Ovenstående er et sæt, der kun indeholder lineære kombinationer, af en ikke-nul-vektor $v$.
Hvis et tomt sæt af 2-vektorer skal beskrives, findes en vektor i sættet, en nul-vektor. Denne vektor er beskrevet som $[0,0]$, i vektor notation.

Samtidig beskriver det ovenstående sæt også, at der for en 2-vektor, findes uendelig mange punkter i eksempelvis $\set*{\alpha[2,3]; \alpha\in\mathbb{R}}$, hvor vektoren kan 

I en spænding af $\set{[1,0],[0,1]}$, er de to vektorer standard for $\mathbb{R}^2$, hvorved hver 2-vektor er i spændingen. Dette betyder at sættet indeholder alle punkter i Den Euklidiske Plan.

\begin{frdef}[Det Euklidiske Rum]
	Det Euklidiske Rum, nogle gange kaldet det Kartesianske rum, eller $\mathbf{n}$-rummet, er rummet, der indeholder alle n-tupler, af reelle tal ($\set{x_1,x_2,\dots,x_n}$). $\mathbb{R}^n$ er derfor et vektorrum, og betegnes også som n-vektorer. Derved har man at $\mathbb{R}^1=\mathbb{R}$, $\mathbb{R}^2$ er Den Euklidiske Plan.
\end{frdef}

