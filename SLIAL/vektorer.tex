Dette kapitel vil beskrive todimensionale vektorer som 2-vektorer, og tredimensionelle vektorer som 3-vektorer. Der vil blive brugt bracket notation for vektorer. Det er her fordelagtigt at introducere Det Euklidiske Rum.
\begin{frdef}[Det Euklidiske Rum]\label{def:euklid_rum}
	Det Euklidiske Rum, nogle gange kaldet det Kartesianske rum, eller $\mathbf{n}$-rummet, er rummet, der indeholder alle n-tupler, af reelle tal ($\set{x_1,x_2,\dots,x_n}$). $\mathbb{R}^n$ er derfor et vektorrum, og betegnes også som n-vektorer. Derved har man at $\mathbb{R}^1=\mathbb{R}$, $\mathbb{R}^2$ er Den Euklidiske Plan.
\end{frdef}

Det fremkommer af Det Euklidiske Rum, at en 4-vektor, kan skrives som $\mathbb{R}^4$. Dette er da Definition \ref{def:euklid_rum} siger, at n-vektorer, skal være $\mathbb{R}^n$.

\begin{frdef}[Vektorer]
	En vektor er i geometrien et objekt, der er defineret ved at have en længde og en retning.
\end{frdef}


\section{Geometrien i sæt af vektorer}
Dette afsnit har til formål at beskrive det geometriske aspekt af vektorer. Det vil dække spænding af vektorer. 

\subsection{Spændinger af vektorer over $\mathbb{R}$}
 
\begin{equation}
	\mathrm{Span}\{v\}=\set*{\alpha v; \alpha\in\mathbb{R}}
\end{equation}

Ovenstående er et sæt, der kun indeholder lineære kombinationer, af en ikke-nul-vektor $v$.
Hvis et tomt sæt af 2-vektorer skal beskrives, findes en vektor i sættet, en nul-vektor. Denne vektor er beskrevet som $[0,0]$, i vektor notation.

Samtidig beskriver det ovenstående sæt også, at der for en 2-vektor, findes uendelig mange punkter i eksempelvis $\set*{\alpha[2,3]; \alpha\in\mathbb{R}}$, hvor vektoren kan 

I en spænding af $\set{[1,0],[0,1]}$, er de to vektorer standard for $\mathbb{R}^2$, hvorved hver 2-vektor er i spændingen. Dette betyder at sættet indeholder alle punkter i Den Euklidiske Plan.

En 4-vektor kan også betegnes som et sæt af elementer eller en funktion. Betragt følgende vektor $\set{2.87,6.2,3.7,1.5}$. Dette kan betegnes som funktionen
\begin{align}
	\label{eqn:vec_func}
	0&\mapsto 2.87\\
	1&\mapsto 6.2\\
	2&\mapsto 3.7\\
	3&\mapsto 1.5
\end{align}

Samme 4-vektor kan repræsenteres som en Python Dictionary $\set{0:2.87, 1:6.2, 2:3.7, 3:1.5}$.

\subsection{Sparsitet}
En vektor hvis værdier alle er 0, kaldes en sparsom eller nul-vektor. Hvis ikke mindre end $k$ værdier i vektoren er 0, kaldes vektoren en k-sparsom vektor. Vektorer der repræsenterer data, der er opsamet af sensorer, er næste aldrig sparsomme vektorer.

\section{Hvad kan vises med vektorer}
Mange ting kan vises med vektorer. Blandt disse ting er \emph{Binære strenge}, altså strænge at 1 og 0, der har en given værdi. Disse vises ved vektorer som $\set{1,0,0,1,1,0,1}$, hvor vektoresn længde, er det samme som længden af den binære streng. \emph{Attributer} kan også vises med vektorer. Dette er som vist i \cref{eqn:vec_func}, hvor to værdier kan sidestilles som en Python Dictionary. Mere relevant for sandsynlighedsteori, kan vi også repræsenterer \emph{Sandsynligheds distribution} med vekoterer. Dette vises igen som i \cref{eqn:vec_func}, hvor hver key er mapped til en værdi. \emph{Billeder} kan også vises som vektorer, da hvor farve har en RGB værdi, som kan afkodes til en binær værdi. Ydermere kan punkter i rummet plottes som vektorer, dog er dette givet ved Mat A adgangskursus, så dette behøver ikke yderligere forklaring.

\section{Vektor addition}
Vektor addition foregår som på Mat A adgangskurset. Dog vil denne sektion gå mere i dybten med vektor addition. 

\begin{frdef}[KLEIN s. 69]
	Addition af n-vekoterer er definieret ved addition af de respektive værdier.
	$[u1,u2,u3] + [v1,v2,v3] = [u1+v1, u2+v2, u3+v3]$
\end{frdef}

Siden vektorer som $[1,2]$ kan bruges som et opslag, kan den tænkes som at overføre en værdi. Eksempelvis fra $[4,4]$ til $[5,6]$. Derfor må den også kunne adderes.

\begin{frdef}[Generering af vektorer]
	Lad V være et sæt af vektorer. Hvis $v_1,\dots,v_n$ er vektorer, således at $V=\set{v_1,\dots,v_n}$
\end{frdef}