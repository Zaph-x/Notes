\section{Vektorrum}
Først defineres $\mathcal{V}$ som et Vektorrum.
$\mathcal{V}$ har følgende egenskaber:
\begin{enumerate}
    \item $\mathcal{V}$ indeholder en nul-vektor
    \item For hver vektor $v$, hvis $\mathcal{V}$ indeholder $v$, indeholder den også $\alpha v$ for hver scalar $\alpha$
    \item For hvert par $u$ og $v$ af vektorer, hvis $\mathcal{V}$ indeholder $u$ og $v$, indeholder dette også $u+v$
\end{enumerate}
For ovenstående tre egenskaber kaldes disse V$n$.
\begin{frdef}
    Et sæt $\mathcal{V}$ af vektorer kaldes et vektorrum, hvis det overholder ovenstående tre egenskaber.
\end{frdef}
Udtrykket i V2 udtrykkes matematisk som ``$\mathcal{V}$ is \textit{closed under} scalar-vector multiplication.''
Ligeledes udtrykkes V3 som ``$\mathcal{V}$ is \textit{closed under} addition.''
\begin{frdef}
    Et vektorrum der kun består af en nul-vektor, kaldes et \textit{trivielt} vektorrum.
\end{frdef}

\subsection{Subrum}
\begin{frdef}
    Hvis $\mathcal{V}$ og $\mathcal{W}$ er vektorrum, og $\mathcal{V}$ er et subset af $\mathcal{W}$, siges det om $\mathcal{V}$ at være et Subrum af $\mathcal{W}$.
\end{frdef}
Det kan derfor vises at sættet $\set{[0,0]}$ (Det trivielle vektorrum), er et subset af $\set{\alpha[2,1];\alpha\in\mathbb{R}}$ og dermed også et subset af $\mathbb{R}^2$.
Ydermere er $\mathbb{R}^2$ det største subset af $\mathbb{R}^2$.

\subsection{Affine vektorrum}
Det afine vektorrum indeholder vektorer, der ikke går igennem oregon.
Dette skrives matematisk som $\set{a+v;v\in\mathcal{V}}$.
Dette kan omskrives til $a++\mathcal{V}$.

