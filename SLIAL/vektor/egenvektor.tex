\section{Egenvektor}
En egenvektor er findes, hvis vektor-matrice produktet ikke er 0, efter multiplikation.
For fælgende eksempel findes det at der er tale om en egenvektor, da produktet ikke er 0:
\begin{equation}
    \left[
        \begin{matrix}
            1\\
            0\\
            -1
        \end{matrix}
    \right]
    \cdot
    \left[
        \begin{matrix}
            -1&-1&3\\
            -1&3&-1\\
            3&-1&-1
        \end{matrix}
    \right]
    =
    \left[
        \begin{matrix}
            -4\\
            0\\
            4
        \end{matrix}
    \right]
\end{equation}
Da produktet ikke er en nulvektor, kan det konkluderes at vektoren, der ganges på, er en egenvektor.
