\section{x64 Assembly}
\subsection{Registre}
x64 Assembly kode har 16 64-bit registre at gøre godt med. 
Disse 16 registre kan ydermere opdeles i 32- 16- og 8-bit registre, som vist i \cref{tab:registers}
\begin{table}[h]
    \centering
    \begin{tabular}{c|ccc}
        8-byte register&Bytes 0-3&Bytes 0-1&Byte 0\\\hline
        \verb|%rax|&\verb|%eax|&\verb|%ax|&\verb|%al|\\
        \verb|%rcx|&\verb|%ecx|&\verb|%cx|&\verb|%cl|\\
        \verb|%rdx|&\verb|%edx|&\verb|%dx|&\verb|%dl|\\
        \verb|%rbx|&\verb|%ebx|&\verb|%bx|&\verb|%bl|\\
        \verb|%rsi|&\verb|%esi|&\verb|%si|&\verb|%sil|\\
        \verb|%rdi|&\verb|%edi|&\verb|%di|&\verb|%dil|\\
        \verb|%rsp|&\verb|%esp|&\verb|%sp|&\verb|%spl|\\
        \verb|%rbp|&\verb|%ebp|&\verb|%bp|&\verb|%bpl|\\
        \verb|%r8|&\verb|%r8d|&\verb|%r8w|&\verb|%r8b|\\
        \verb|%r9|&\verb|%r9d|&\verb|%r9w|&\verb|%r9b|\\
        \verb|%r10|&\verb|%r10d|&\verb|%r10w|&\verb|%r10b|\\
        \verb|%r11|&\verb|%r11d|&\verb|%r11w|&\verb|%r11b|\\
        \verb|%r12|&\verb|%r12d|&\verb|%r12w|&\verb|%r12b|\\
        \verb|%r13|&\verb|%r13d|&\verb|%r13w|&\verb|%r13b|\\
        \verb|%r14|&\verb|%r14d|&\verb|%r14w|&\verb|%r14b|\\
        \verb|%r15|&\verb|%r15d|&\verb|%r15w|&\verb|%r15b|
    \end{tabular}
    \caption{Oversigt over registre i x64 Assembly}
    \label{tab:registers}
\end{table}

\subsection{Ord og bytes}
Indenfor Assembly arbejdes der med ord og bytes.
\begin{itemize}
    \item En \textit{byte} er 8 bits ($10011011$).
    \item Et \textit{word} er 2 bytes ($10011011\:10010110$).
    \item Et \textit{dword} er 4 bytes og står for double word.
    \item Et \textit{qword} er 8 bytes og står for quad word.
\end{itemize}
Bogstaverne \textit{d} og \textit{q} indgår også til tider i operationer, som eksempelvis \textit{movq}.
Dette betyder blot at der flyttes noget, der er 8 byte stort.
\subsection{Instruktioner}