\documentclass[11pt, twoside, a4paper, openright]{report}         % This sets up the documentclass with font size and paper size already set
\usepackage[utf8]{inputenc}                                     % This will make the document use UTF-8 characters
\usepackage[danish]{babel}                                      % Changing the language to English, if this is not the default
\usepackage[T1]{fontenc}                                        % The font encoding



%%%%%%%%% GRAPHICS %%%%%%%%%%

\usepackage{graphics,graphicx}
\usepackage{epstopdf}
\usepackage{listings}                                           % OPTIONAL - This will allow for code listings in the report
\usepackage{tabularx}                                           % Allows for tables in the report
\usepackage{array, booktabs}                                    % Better table formatting
\usepackage{color}                                              % Allows for use of colours
\usepackage{xcolor}                                             % More colours

\usepackage{caption}                                            % Allows for custom cations
\captionsetup{font=footnotesize, labelfont=bf}                  % Sets the font size in the custom captions, as well as font type to bold

\usepackage{framed}                                             % Allow frames around listings and theorems

%%%%%%%% OPTIONAL LISTING MODIFIERS %%%%%%%%%

%%%%%%%% CODE %%%%%%%%%%%

% Adds code snippet functionality
\usepackage{listings}
\renewcommand{\lstlistingname}{Kodeuddrag}
\usepackage{tcolorbox}

\definecolor{commentGreen}{RGB}{34,139,24}
\definecolor{stringPurple}{RGB}{208,76,239}
\definecolor{keywordPurple}{RGB}{168,26,143}
\definecolor{indentifierPurple}{RGB}{91,44,151}
\definecolor{stringRed}{RGB}{194,29,32}
\definecolor{commentGreen}{RGB}{22,150,25}
\usepackage{tgheros}

\lstset{language=[Sharp]C,
  caption={Uddrag af C\# program},
  label=DescriptiveLabel,
  numbers=left, numberstyle=\tiny,
  backgroundcolor=\color{black!1},
}

\lstset{language=Python,
  caption={Python kode udsnit},
  label=DescriptiveLabel,
  numbers=left, numberstyle=\tiny,
  xleftmargin=\parindent,
  backgroundcolor=\color{black!1},
  basicstyle=\footnotesize\ttfamily,
  keywordstyle=\bfseries\color{keywordPurple},
  commentstyle=\itshape\color{stringRed},
  identifierstyle=\color{indentifierPurple},
  stringstyle=\color{commentGreen},
}

\lstdefinestyle{customcsharp}{
  belowcaptionskip=1\baselineskip,
  breaklines=true,
  frame=tlrb,
  rulecolor=\color{black!20},
  xleftmargin=\parindent,
  language=[Sharp]C,
  showstringspaces=false,
  basicstyle=\footnotesize\ttfamily,
  keywordstyle=\bfseries\color{keywordPurple},
  commentstyle=\itshape\color{stringRed},
  identifierstyle=\color{indentifierPurple},
  stringstyle=\color{commentGreen},
}

\lstdefinestyle{custonpython}{
  belowcaptionskip=1\baselineskip,
  breaklines=true,
  frame=tlrb,
  rulecolor=\color{black!20},
  xleftmargin=\parindent,
  language=Python,
  showstringspaces=false,
  basicstyle=\footnotesize\ttfamily,
  keywordstyle=\bfseries\color{keywordPurple},
  commentstyle=\itshape\color{commentGreen},
  identifierstyle=\color{indentifierPurple},
  stringstyle=\color{stringRed},
}

\lstdefinestyle{customasm}{
  belowcaptionskip=1\baselineskip,
  frame=L,
  xleftmargin=\parindent,
  language=[x86masm]Assembler,
  basicstyle=\footnotesize\ttfamily,
}

\lstset{escapechar=``,style=customcsharp}
\definecolor{hisyntxcolor}{rgb}{0.95, 0.95, 0.96}
\newtcbox{\hisyntx}{nobeforeafter, colback=hisyntxcolor!80,boxrule=0pt,boxsep=0pt,left=4pt,right=4pt,top=3.5pt,bottom=3.5pt,tcbox raise base}

%%%%%%%% MATHEMATICS %%%%%%%%

\usepackage{amsmath}                                            % Allign correctly
\usepackage{amssymb}                                            % More mathematica symbols
\usepackage{amsthm}                                             % Theorems
\usepackage{mdframed}
\usepackage[amssymb]{SIunits}                                   % Support SI units
\usepackage{units}                                              % Typeset units correct
\usepackage{amsfonts}                                           % Math fonts   
\usepackage{mathtools}      
\usepackage{xparse}  
%\usepackage[framed,amsmath,thmmarks]{ntheorem}                  % Allows theorems. With amsmath you can also you \eqref{label} to reference an equation

\newtheoremstyle{note}% <name>
{3pt}% <Space above>
{3pt}% <Space below>
{}% <Body font>
{}% <Indent amount>
{\itshape\bfseries\large}% <Theorem head font>
{.\newline}% <Punctuation after theorem head>
{.5em}% <Space after theorem headi>
{}% <Theorem head spec (can be left empty, meaning `normal')>
\theoremstyle{note}
\newtheorem{frtheo}{Sætning}
\newtheorem{frdef}{Definition}
\newtheorem{frlem}{Lemma}
\newtheorem{frex}{Eksempel}
\surroundwithmdframed[backgroundcolor=blue!10,linecolor=blue!10]{frtheo}
\surroundwithmdframed[backgroundcolor=blue!10,linecolor=blue!10]{frdef}
\surroundwithmdframed[backgroundcolor=blue!10,linecolor=blue!10]{frlem}
\surroundwithmdframed[backgroundcolor=blue!10,linecolor=blue!10]{frex}
\usepackage{tikz,tkz-graph,tkz-berge}
\renewcommand*{\VertexLineWidth}{1pt}%vertex thickness
\renewcommand*{\EdgeLineWidth}{1pt}% edge thickness

%%%%%%% LAYOUT %%%%%%%%

\usepackage[inner=28mm,outer=41mm,]{geometry}                   % Corrects the margins
\usepackage{titlesec}                                           % Prettier titles
\titleformat{\chapter}[hang]{                                   % Format the chapter title
    \normalfont\Huge\bfseries                                   % Format the chapter title
}{    \chaptertitlename\ \thechapter}{20pt}{\Huge}              % Format the chapter title
\titleformat*{\section}{\normalfont\Large\bfseries}             % Format section titles
\titleformat*{\subsection}{\normalfont\large\bfseries}          % Format subsection titles
\titleformat*{\subsubsection}{\normalfont\normalsize\bfseries}  % Format subsubsection titles

\usepackage{csquotes}
\usepackage{fancyhdr}                                           % Fancy headers
\pagestyle{fancy}                                               % Set the pagestyle to being fancy. This will write the chapter and page number to the header
\setlength{\headheight}{13.6pt}
\fancyhf{}                                                      % Clear the header and footer
\fancyhead[RE]{\small\nouppercase\leftmark}                     % Even pages have the chapter title
\fancyhead[LO]{\small\nouppercase\rightmark}                    % Odd pages have the section title
\fancyhead[LE,RO]{\thepage}                                     % Pagenumber on every page
\raggedbottom                                                   % Do not streach content. Instead replace it with a whitespace

\renewcommand{\headrulewidth}{0pt}                              % OPTIONAL Remove the header line seperating header and page

\usepackage{calc}                                               % Better layout formatting. Allows for arithmetics with length and such

%%%%%%% BIBLIOGRAPHY %%%%%%%%%

\usepackage[                                                    % Format of bibliography
    backend=bibtex,                                             % Use bibtex as backend manager
    style=nature,                                               % Style the bibliography right (See https://www.overleaf.com/learn/latex/Biblatex_citation_styles for more styles)
    bibencoding=utf8                                            % Encode bibliography with UTF-8 charset
]{biblatex}                                                     % Format of bibliography                           
%\bibliography{bib/mybib}

%%%%%%% GOOD TO HAVES %%%%%%%%%

\usepackage[nottoc]{tocbibind}                                  % Add bibliography to table of contents
\usepackage{lastpage}                                           % Allows you to reference the last page of the paper
\usepackage[                                                    % Adds todo notes to the document
    textwidth=\marginparwidth,                                  % Sets the text width og the todo notes as the width of the margin
    textsize=scriptsize                                         % Sets the fint size to a smaller size for more compact notes
]{todonotes}                                                    % Adds todo notes to the document

%%%%%%% REFERENCING AND LINKS %%%%%%%%

\usepackage{nameref}                                            % Allows you to refer to sections by name
\usepackage{hyperref}                                           % Enable hyperlinks and other referencing methods
\usepackage[danish]{cleveref}                                           % Better figure, table and section referencing
%\crefname{appendix}{bilag}{bilag}
%\crefname{subappendix}{bilag}{bilag}
%\crefname{subsubappendix}{bilag}{bilag}
%\crefname{section}{afsnit}{afsnit}
%\crefname{subsection}{afsnit}{afsnit}
%\crefname{subsubsection}{afsnit}{afsnit}
%\crefname{chapter}{kapitel}{kapitel}
\crefname{figure}{Figur}{Figur}
\crefname{table}{Tabel}{Tabel}
\crefname{equation}{Ligning}{Ligning}
\crefname{tikzpicture}{Graf}{Graf}
\usepackage{nameref}                                            % Load again to fix compatibility issues

\hypersetup{                                                    % Set up the pdf with correct data
	plainpages=false,                                           
	pdfauthor={Author(s)},                                      % Here you put your names
	pdftitle={Title},                                           % This should by default save it with the correct title
	pdfsubject={Subject},                                       % The subject of the report
	bookmarksnumbered=true,
	colorlinks=false,                                           % Are links coloured?
	citecolor=black,                                            % Colour of the citations
	filecolor=black,                                            % Colour of files
	linkcolor=black,                                            % Colour of links
	urlcolor=black,                                             % Colour of URLs
	pdfstartview=FitH                                           % How should the PDF scale when loaded into viewer
}

\DeclarePairedDelimiterX{\set}[1]{\{}{\}}{\setargs{#1}}
\NewDocumentCommand{\setargs}{>{\SplitArgument{1}{;}}m}
{\setargsaux#1}
\NewDocumentCommand{\setargsaux}{mm}
{\IfNoValueTF{#2}{#1} {#1\nonscript\:\delimsize\vert\allowbreak\nonscript\:\mathopen{}#2}}%
\def\Set{\set*}%