\subsection{Throughput og Latency}
I CART måles \textit{Circuit delay} som picosekunder (ps).
I et system, hvor der eksisterer 1 blok computationel logik, med et delay på 300ps, og et register load på 20ps, gives et throughput på 3.12GIPS.
Dette kan udregnes på formlen:
\begin{equation}
    Throughput=\frac{instruktion}{delay}\cdot\frac{1000ps}{1ns}
\end{equation}
Throughput angives i GIPS \textit{(Giga-instruktioner per sekund)}, og skal derfor ganges med 1000.
Den fulde tid det tager at køre en beregning fra start til slut, kaldes latency, og er i ovenstående eksempel 320ps.
Latency i \textit{n-stage} kan udregnes ved følgende formel:
\begin{equation}
    Latency=clock\;cycler\cdot instruktion\;delay
\end{equation}
Med disse formler kan en forbedring og ny latency beregnes.
Dette gøres ved brug af følgende to formler:
\begin{equation}
    \Delta_{Throughput}= \frac{Throughput_{new}}{Throughput_{old}}
\end{equation}
\begin{equation}
    \Delta_{Latency}=\frac{Latency_{new}}{Latency_{old}}
\end{equation}
Den øgede latency skyldes at der er tilføjet mere hardware, og flere pipeline registre.
For bedre forståelse se lærebogen s. 451.
\begin{table}[h!]
    \centering
    \begin{tabular}{ll}
        \hline
        Unit&Seconds\\\hline
        1000ps&$1e^{-9}$\\
        1ns&$1e^{-9}$\\\hline
    \end{tabular}
    \caption{Convertering af Picosekunder og nanosekunder}
\end{table}